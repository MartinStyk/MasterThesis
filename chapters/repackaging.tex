\chapter{Prebalené aplikácie}
\label{chap:repackaging}
Prebaľovanie aplikácií je formou útoku na systému Android. Princíp útoku spočíva v~modifikácii originálnej aplikácie a jej následnej redistribúcii. Počas modifikácie APK súboru môže útočník pridať škodlivú funkcionalitu a upraviť tak správanie aplikácie.
Prebaľovanie aplikácií je jednou z~najbežnejších techník distribúcie malvéru a škodlivého kódu na platforme Android. 

Štandardný postup tvorby prebalených aplikácií z~pohľadu útočníka:
\begin{enumerate}
	\item Stiahnutie populárnej aplikácie,
	\item Dekompilácia aplikácie pomocou nástrojov reverzného inžinierstva,
	\item Modifikácia aplikácie pridaním škodlivého kódu,
	\item Zabalenie aplikácie,
	\item Podpísanie vlastným self-signed certifikátom,
	\item Distribúcia aplikácie pomocou oficiálnych a neoficiálnych kanálov.
\end{enumerate}

Tvorcovia prebalených aplikácií využívajú popularitu originálnych aplikácií za účelom propagácie malvéru. Cieľom prebaľovania APK súborov nie je vytvoriť novú odlišnú aplikáciu s~využitím zdrojového kódu a vlastností existujúcej aplikácie. Namiesto toho je zámerom prebalených aplikácií vydávať sa za pôvodnú aplikáciu a pripomínať ju v~maximálnej možnej miere. Modifikované verzie navonok pôsobia identicky ako originálne verzie. Z~pohľadu užívateľa poskytuje modifikovaná aplikácia rovnakú funkcionalitu a užívateľské rozhranie ako originál. 

Prebaľovanie aplikácií je populárne aj vďaka warezovým portálom slúžiacich na zdieľanie nelegálneho obsahu. Warezové portály ponúkajú bezplatne aj aplikácie, ktoré sú v~službe \zv{Google Play} dostupné iba za poplatok, čo výrazne prispieva k~šíreniu takýchto aplikácií. Prebalené aplikácie však nie sú distribuované výhradne pomocou neoficiálnych kanálov, ale aj prostredníctvom \zv{Google Play}.

Prebalené aplikácie sú škodlivé nie len pre koncových užívateľov, ale aj vývojárov, vlastníkov duševných práv a prevádzkovateľov obchodov s~aplikáciami~\cite{CloneRelative}. 

\section{Spôsob modifikácie aplikácií}
Za účelom modifikácie aplikácie upraví útočník inštalačný APK balíček. Modifikácia už existujúcej nainštalovanej aplikácie nie je možná. Spravidla je modifikovaný len jej inštalačný balíček a škodlivá aplikácia sa do zariadenia dostane inštaláciou takéhoto balíčku.
APK súbory majú definovaný formát a štruktúru, ktoré sú popísané v~kapitole \ref{sec:struktura}. Vďaka ich formátu a štrukturovanému procesu tvorby je ich dekompilácia nenáročná.

Existuje viacero nástrojov reverzného inžinierstva, ktoré môžu byť použité pri tvorbe prebalených aplikácií. Najkomplexnejším nástrojom, ktorý je možné použiť na kompletné rozbalenie a následné zabalenie aplikácie je nástroj ApkTool~\cite{Apktool}. Za účelom modifikácie zdrojového kódu je možné využiť bežne dostupné nástroje ako \zv{dex2jar}, \zv{jd-gui} a \zv{smali} ~\cite{Dex2jar, jdgui, smali}.

Z~hľadiska modifikácie sú pri prebaľovaní zaujímavé nasledujúce súbory:

\begin{itemize}
	\item \bod{AndroidManifest.xml} -- upravením tohto súboru je možné pridávať nové bezpečnostné povolenia. Súbor je v~APK balíku dostupný ako skomprimované XML. Skomprimované (binárne) XML je možné konvertovať do editovatelnej formy pomocou nástroja \zv{ApkTool}~\cite{Apktool}.
	\item \bod{Classes.dex} -- vykonanie zmien v~zdrojovom kóde umožňuje komplentú modifikáciu funkcionality aplikácie. Tento súbor obsahujúci skompilovaný zdrojový kód aplikácie je možné editovať pomocou spomínaných nástrojov \zv{dex2jar}, \zv{jd-gui} a \zv{smali}.
\end{itemize}

Zásadnou vecou, ktorá odlišuje prebalenú aplikáciu od originálu je digitálny podpis. Po editácii aplikácie a jej opätovnom zabalení je nutné aplikáciu digitálne podpísať. Za predpokladu, že privátny kľúč pôvodného vydavateľa neunikol a nie je k~dispozícii útočníkovi, podpis originálnej a prebalenej aplikácie sa bude líšiť. 

\section{Časté typy modifikácie}

Modifikácia zdrojových kódov aplikácie predstavuje vážne ohrozenie bezpečnosti. Nasledujúce body popisujú najčastejšie typy modifikácie zdrojových kódov.

\subsection*{Modifikácia reklamných modulov}
Väčšina aplikácií je užívateľom dostupná zadarmo. Zdroj príjmu pre vývojárov predstavujú reklamné moduly integrované do aplikácie. Najčastejšou modifikáciou vykonávanou pri prebaľovaní aplikácií je nahradenie pôvodných reklamných modulov. Výnosy z~reklamy sú tak presmerované z~účtu pôvodného vývojára na účet útočníka. Najjednoduchšou úpravou je využitie existujúcich reklamných služieb a presmerovanie zisku na iný účet pomocou zmeny API kľúča. Takmer 65 \% prebalených aplikácií obsahuje modifikované reklamné knižnice~\cite{fakeapps}.

\subsection*{Zneužitie prémiových služieb}
Niektoré prebalené aplikácie obsahujú platené služby, ktoré nie sú súčasťou originálnej aplikácie. Najčastejším spôsobom zneužitia prémiových služieb je odosielanie textových správ na spoplatnené prémiové čísla. Prebalená aplikácia vyžaduje pri inštalácii povolenie na odosielanie správ, ktoré ďalej odosiela bez používateľovho vedomia. Takto upravená aplikácia môže zaregistrovať vlastnú službu, ktorá odfiltruje SMS správy od telekomunikačného poskytovateľa a zabráni tak upozorneniam o~využití platených prémiových služieb. Známym príkladom takto napadnutej aplikácie je prehliadač QQ Browser~\cite{fakeapps}.

\subsection*{Prístup k~citlivým dátam}
Aplikácie môžu byť upravené za účelom zbierania informácií ako telefónne čísla kontaktov, emailové adresy, história prehliadača alebo zoznam nainštalovaných aplikácií. Túto činnosť vykonávajú služby na pozadí. Zozbierané dáta sú odosielané na vzdialené webové servre. 

\subsection*{Inštalácia malvéru}
Prebalené aplikácie môžu na pozadí spustiť sťahovanie a inštaláciu ďalšieho malvéru. 

\subsection*{Vzdialený prístup a ovládanie zariadenia}
Za účelom získania kontroly nad zariadením môže aplikácia nadviazať spojenie so vzdialeným serverom, ktorý je pod kontrolou útočníka. Server posiela aplikácii príkazy a prijíma od nej odpovede. Príkladom takéhoto správania je prebalená verzia aplikácie Stupid Birds, ktorá obsahuje vložený kód, pomocou ktorého sa pripojí na URL adresu z~ktorej prijíma príkazy. Na základe týchto príkazov dokáže aplikácia na pozadí stiahnúť iné aplikácie, a umiestniť odkazy na ich inštaláciu priamo na plochu zariadenia~\cite{fakeapps}. 

\subsection*{Vloženie viacerých DEX súborov}
Po zmenách aplikácie prezentovaných v~predchádzajúcich bodoch je nutné APK balíček nanovo podpísať. Vo verziách Android 4.4 a nižších existuje zraniteľnosť známa ako Master Key vulnerability. Ide o~jednu z~najzávažnejších zraniteľností pri overovaní APK balíčku počas jeho inštalácie. V~prípade, že APK balíček obsahuje dva súbory s~rovnakým názvom, pri overovaní hashov súborov sa použije prvý záznam o~takomto súbore, avšak Android použije pri inštalácii druhý súbor. To umožňuje upraviť aplikáciu bez potreby opätovného podpísania~\cite{c2gYRVCI9leJhfOJ}. Tento fakt je možné využiť na vloženie nového súboru so zdrojovým kódom, ktorý kompletne nahradí pôvodnú logiku alebo vzhľad aplikácie.

\section{Výskyt prebalených aplikácií}
Pre platformu Android existujú milióny aplikácií. Počet aplikácií dostupných v~oficiálnom obchode \zv{Google Play} prekročil v~septembri 2017 3 milióny~\cite{Statista}. Množstvo aplikácií je dostupných prostredníctvom neoficiálnych distribučných kanálov.  Výsledky predchádzajúcich výskumov ukazujú, že prebalené aplikácie predstavujú nezanedbateľnú časť dostupných aplikácií.

Analýzou 23\,000 aplikácií zo šiestich alternatívnych obchodov v~rámci práce \zv{Detecting Repackaged Smartphone Applications in
Third-Party Android Marketplaces} sa zistilo, že pomer prebalených aplikácií sa v~alternatívnych obchodoch pohybuje medzi 5 až 13 \% ~\cite{DetectingRepackagedZhou}.
Počet aplikácií, ktoré sú nie len prebalené, ale pridávajú do originálnych aplikácií škodlivé správanie je taktiež významný. Wu Zhou a kolektív na základe analýzy vzorky 85\,000 aplikácií identifikovali, že počet takýchto aplikácií sa pohybuje medzi 0,97 až 2,7 \% všetkých aplikácií~\cite{Zhou2013}.
Až 86 \% aplikácií obsahujúcich malvér vzniklo prebalením originálnej aplikácie a vložením škodlivého kódu~\cite{androidThreats}.

Alternatívne distribučné platformy, ktoré často neposkytujú dostatočnú formu kontroly obsahu, obsahujú väčší počet prebalených APK súborov ako oficiálne zdroje~\cite{Zhauniarovich2013}. 
