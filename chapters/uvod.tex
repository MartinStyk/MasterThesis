\chapter*{Úvod}
\addcontentsline{toc}{chapter}{Úvod} Android je globálne najrozšírenejší mobilný operačný systém. Tento systém, pôvodne určený pre chytré mobilné telefóny, sa dnes vďaka širokému spektru poskytovaných služieb a prispôsobiteľnosti rozšíril na viacero nových platforiem, ako napríklad zábavné systémy automobilov, softvér televízií, notebooky alebo inteligentné hodinky. Jedným z~hlavných dôvodov popularity Androidu je dobrá dostupnosť aplikácií. Vďaka veľkému množstvu aplikácií má každý užívateľ možnosť prispôsobiť softvérový obsah zariadenia svojim potrebám. Z~pohľadu užívateľa je kľúčová funkcionalita poskytovaná aplikáciou. Aplikácie však obsahujú aj množstvo metadát, ktoré nám môžu o~ich fungovaní veľa prezradiť. Mobilné aplikácie často pristupujú k~citlivým dátam užívateľov. Preto je dôležité zaistiť bezpečnosť na všetkých úrovniach. Systém Android dlhodobo bojuje s~bezpečnostným rizikom spojeným s~distribúciou aplikácií. Táto zraniteľnosť spočíva v~modifikácii existujúcej aplikácie tak, aby vykonávala aj činnosť ktorá je škodlivá pre užívateľa alebo pôvodného vývojára aplikácie. Takto modifikované aplikácie sú často označované ako prebalené.

Táto práca sa zaoberá získavaním metadát o~aplikáciách na systéme Android a ich následnom využití pri detekcii potenciálne škodlivých prebalených aplikácií.

Úvodná časť práce má za úlohu oboznámiť čitateľa s~hlavnými aspektami Android aplikácií so zameraním na bezpečnosť a spôsob ich distribúcie. Práca predstavuje metódu prebaľovania inštalačných APK súborov spolu s~vybranými spôsobmi detekcie takýchto aplikácií.

Cieľom praktickej časti tejto práce je vytvoriť mobilnú aplikáciu slúžiacu na analýzu aplikácií pre operačný systém Android. Táto aplikácia by mala získavať detailné metainformácie o~nainštalovaných aplikačných balíčkoch. Tieto informácie by mali obsahovať údaje o~zabezpečení a digitálnom podpise, rôznych komponentoch aplikácie a bezpečnostných povoleniach vyžadovaných aplikáciou. Vytvorená aplikácia by mala tieto informácie sprostredkovávať užívateľom v~prehľadnej a zrozumiteľnej forme. Cieľom je publikovať mobilnú aplikáciu v~obchode \zv{Google Play}.

Mobilná aplikácia nebude analyzovať nainštalované aplikácie výhradne za účelom prezentácie metadát užívateľovi. Zámerom tejto práce je využiť metadáta získané analýzou aplikácií za účelom detekcie prebalených kópií. Cieľom je návrhnúť a implementovať metódu detekcie prebalených aplikácií využívajúcu podobnosť ich metadát. Metóda detekcie prebalených aplikácií by sa mala opierať o~zistenia predchádzajúcich výskumov, ktoré ďalej rozširuje a posúva z~vedeckej roviny do praxe. Táto metóda bude sprostredkovaná koncovým užívateľom. Na základe reálne dosiahnutých výsledkov a spätnej väzby od užívateľov budú zhodnotené klady a zápory navrhnutej metódy.

Očakávaným výsledkom práce je kompletný systém na analýzu aplikácií a detekciu prebalených kópií, pozostávajúci z~mobilnej aplikácie a serveru s~databázou, ktorá obsahuje dáta o~veľkom počte aplikácií.

Prvá kapitola obsahuje stručné oboznámenie s~aplikáciami pre platfomu Android a zameriava sa na popis aplikačného balíčka, spôsobu distribúcie, inštalácie a zabezpečenia aplikácií. V~kapitole číslo 2 je predstavený koncept prebalených aplikácií. Nasledujúca kapitola 3 obsahuje popis vybraných metód detekcie prebalených aplikácií. Štvrtá kapitola popisuje vysokoúrovňový návrh systému pre analýzu a detekciu prebalených aplikácií. Piata kapitola obsahuje detailné informácie o~vyvíjanej mobilnej aplikácií. Kapitola číslo 6 sa venuje návrhu, implementácii a zhodnoteniu vlastnej metódy detekcie prebalených aplikácií. 
