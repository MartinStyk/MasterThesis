\chapter*{Úvod}
Android je globálne najrozšírenejší mobilný operačný systém. Tento systém, pôvodne určený pre chytré mobilné telefóny, sa dnes vďaka širokému spektru poskytovaných služieb a prispôsobiteľnosti rozšíril na viacero nových platforiem, ako napríklad zábavné systémy automobilov, softvér televízií, notebooky alebo inteligentné hodinky. Jedným z hlavných dôvodov popularity Androidu je dobrá dostupnosť aplikácií. Vďaka veľkému množstvu aplikácií má každý užívateľ možnosť prispôsobiť softvérový obsah zariadenia svojim potrebám. Z pohľadu užívateľa je kľúčová funkcionalita poskytovaná aplikáciou. Aplikácie však obsahujú aj množstvo metadát, ktoré nám môžu o ich fungovaní veľa prezradiť. Mobilné aplikácie často pristupujú k citlivým dátam užívateľov. Preto je dôležité zaistiť bezpečnosť na všetkých úrovniach. Systém Android dlhodobo bojuje s bezpečnostným rizikom spojeným s distribúciou aplikácií. Táto zraniteľnosť spočíva v modifikácii existujúcej aplikácie tak, aby vykonávala činnosť ktorá je škodlivá pre užívateľa alebo pôvodného vývojára aplikácie. Takto modifikované aplikácie sú často označované ako prebalené.

Táto práca sa zaoberá získavaním metadát o aplikáciách na systéme Android a ich následnom využití pri detekcii potenciálne škodlivých prebalených aplikácií.

Teoretická časť práce popisuje hlavné aspekty Android aplikácií so zameraním na bezpečnosť a spôsob distribúcie. Práca predstavuje metódu prebaľovania inštalačných APK súborov spolu s vybranými spôsobmi detekcie takýchto aplikácií.

V rámci praktickej časti je vytvorená mobilná aplikácia, ktorá analyzuje aplikácie pre operačný systém Android. Aplikácia sa zameriava na získanie detailných metainformácií o nainštalovaných aplikačných balíčkoch. Tieto informácie obsahujú údaje o zabezpečení a digitálnom podpise, rôznych komponentoch aplikácie a bezpečnostných povoleniach vyžadovaných aplikáciou.

Práca obsahuje návrh a implementáciu metódy detekcie prebalených APK súborov, založenej na dátach získaných pomocou ich analýzy. Metóda detekcie prebalených aplikácií sa opiera o zistenia predchádzajúcich výskumov, ktoré ďalej rozširuje a posúva z vedeckej roviny do praxe. Prebalené aplikácie sú detekované na základe veľkej kolekcie aplikačných metainformácií. Na tvorbe tejto kolekcie sa podieľajú užívatelia vyvinutej mobilnej aplikácie.  

Výsledkom práce je kompletný systém na analýzu aplikácií a detekciu prebalených kópií. Tento systém pozostáva z mobilnej aplikácie a serveru s databázou, ktorá obsahuje dáta o desaťtisícoch aplikácií. Mobilná aplikácia získavajúca metadáta o aplikáciách je dostupná v obchode \zv{Google Play}. Samotná detekcia prebalených aplikácií prebieha na strane servera. Táto funkcionalita je dostupná prostredníctvom mobilnej aplikácie.

