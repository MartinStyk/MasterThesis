\chapter{Záver}
Primárny cieľom práce bolo vytvorenie softvérového riešenia na detekciu prebalených aplikácií pre mobilný operačný systém Android. Cieľom práce bolo sprístupniť metódu detekcie potenciálne škodlivých aplikácií užívateľom prostredníctvom mobilnej aplikácie, ktorá poskytuje detailné informácie o nainštalovaných aplikačných balíčkoch.  
Za účelom dosiahnutia tohto cieľa sa táto práca zaoberala dvomi hlavnými témami. Prvou z nich bol vývoj mobilnej aplikácie, ktorá získava metadáta o ostatných aplikáciách. Druhú tému predstavuje metóda detekcie prebalených aplikácií. 

V rámci tejto práce bola vyvinutá mobilná aplikácia \zv{Apk Analyzer}. Táto aplikácia poskytuje užívateľovi detailné informácie o aplikačných balíčkoch nainštalovaných na jeho zariadení. Na rozdiel od ostatných dostupných aplikácií ponúkajúcich podobnú funkcionalitu, umožňuje naša aplikácia analýzu APK súborov, ktoré nie sú nainštalované. Aplikácia taktiež poskytuje zrozumiteľné vysvetlenie prezentovaných dát a ponúka užívateľom možnosť detekcie prebalených APK súborov. Aplikácia \zv{Apk Analyzer} (\zv{sk.styk.martin.apkanalyzer}) je dostupná prostredníctvom obchodu \zv{Google Play}. Od prvého vydania aplikácie v októbri 2017 si ju nainštalovalo viac ako 10\,000 (TODO) užívateľov. Kvalitu aplikácie dokumentuje jej hodnotenie v službe \zv{Google Play}. Priemerné hodnotenie dosahovalo v februári 2018 hodnotu 4,7 hviezdičiek (maximum je 5) (TODO update).

V práci bola navrhnutá nová metóda detekcie prebalených aplikácií. Metóda je založená na základoch aktuálnych poznatkov, ktoré odhaľujú, že pri prebaľovaní zvyčajne zostávajú zdrojové súbory nezmenené. Na rozdiel od väčšiny existujúcich prístupov je naša metóda schopná identifikovať, ktorá z aplikácií je originálna a ktoré sú prebalené kópie. Táto metóda pracuje s dynamicky sa vyvíjajúcou databázou metadát o aplikáciách. Na tvorbe tejto databázy sa podieľajú užívatelia mobilnej aplikácie \zv{Apk Analyzer}. Navrhnutá metóda detekcie prebalených APK súborov bola prakticky implementovaná a nasadená v produkčnom prostredí systému \zv{Apk Analyzer}. Od svojho spustenia v TODO zaznamenala TODO požiadavkov na detekciu prebalených aplikácií, z toho TODO aplikácií bolo vyhodnotených ako nebezpečné.


Spojením mobilnej aplikácie a serverovej časti vzniklo softvérové riešenie, umožňujúce detekciu prebalených aplikácií priamo v Android zariadení. Obidve tieto časti boli nasadené v produkčnom prostredí a sú dostupné užívateľom. Zdrojový kód aplikácií vytvorených v rámci tejto práce je voľne prístupný.
