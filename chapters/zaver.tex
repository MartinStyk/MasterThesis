\chapter*{Záver}
\addcontentsline{toc}{chapter}{Záver} 
Táto diplomová práca sa zaoberala analýzou aplikácií nainštalovaných na zariadeniach s~operačným systémom Android. Práca sa zameriavala predovšetkým na získavanie metadát o~aplikáciách a ich využití pri detekcii prebalených aplikácií.

Primárny cieľom práce bolo vytvorenie softvérového riešenia na detekciu prebalených aplikácií pre mobilný operačný systém Android. Cieľom práce bolo sprístupniť metódu detekcie potenciálne škodlivých aplikácií užívateľom prostredníctvom mobilnej aplikácie, ktorá poskytuje detailné informácie o~nainštalovaných aplikačných balíčkoch.  
Za účelom dosiahnutia tohto cieľa sa práca zaoberala tromi hlavnými témami. Prvá časť práce popisuje problematiku prebalených aplikácií a existujúce riešenia ich detekcie. Druhá časť sa zameriava na vývoj mobilnej aplikácie, ktorá získava metadáta o~ostatných aplikáciách. Tretiu tému predstavuje návrh a implementácia metódy detekcie prebalených aplikácií. 

V~rámci tejto práce bola vyvinutá mobilná aplikácia \zv{Apk Analyzer}. Táto aplikácia poskytuje užívateľovi detailné informácie o~aplikačných balíčkoch nainštalovaných na jeho zariadení. Na rozdiel od ostatných dostupných aplikácií ponúkajúcich podobnú funkcionalitu, umožňuje naša aplikácia analýzu APK súborov, ktoré nie sú nainštalované. \zv{Apk Analyzer} taktiež zobrazuje detaily bezpečnostných povolení a zoznamy aplikácií, ktoré dané povolenia vyžadujú. Aplikácia umožňuje užívateľom získať prehľad o~základných štatistických informáciách o~kolekcii aplikácií nainštalovaných na ich zariadení. Aplikácia taktiež poskytuje vysvetlenie prezentovaných dát a ponúka užívateľom možnosť detekcie prebalených APK súborov. Aplikácia \zv{Apk Analyzer} (\zv{sk.styk.martin.apkanalyzer}) je dostupná prostredníctvom obchodu \zv{Google Play}. Od prvého vydania aplikácie v~októbri 2017 si ju nainštalovalo viac ako 30\,000 užívateľov. Kvalitu aplikácie dokumentuje jej hodnotenie v~službe \zv{Google Play}. Aplikáciu v~odbobí do apríla 2018 ohodnotilo 186 užívateľov a dosiahla priemerné hodnotenie 4,6 hviezdičiek (maximum je 5).

V~práci bola navrhnutá nová metóda detekcie prebalených aplikácií. Metóda je založená na základoch aktuálnych poznatkov, ktoré odhaľujú, že pri prebaľovaní zvyčajne zostávajú zdrojové súbory nezmenené. Navrhnutá metóda identifikuje prebalené aplikácie na základe zhody obrázkových súborov medzi viacerými aplikačnými balíčkami. Na rozdiel od väčšiny existujúcich prístupov je naša metóda schopná identifikovať, ktorá z~aplikácií je originálna, a ktoré sú prebalené kópie. Táto metóda pracuje s~dynamicky sa vyvíjajúcou databázou metadát o~aplikáciách. Na tvorbe tejto databázy sa podieľajú užívatelia mobilnej aplikácie \zv{Apk Analyzer}. Po schválení užívateľom odošle mobilná aplikácia metadáta o~všetkých užívateľkých aplikáciách prítomných na danom zariadení. Navrhnutá metóda detekcie prebalených APK súborov bola prakticky implementovaná a nasadená v~produkčnom prostredí systému \zv{Apk Analyzer}.
Vytvorená databáza obsahujúca všetky metadáta obsahuje údaje o~viac ako 600\,000 aplikáciách. Počas reálneho nasadenia metódy detekcie prebalených súborov bolo identifikovaných niekoľko skutočností, ktoré môžu negatívne ovplyvniť výsledok detekcie, napríklad aplikácie využívajúce minimum vlastných obrázkov, obrázky pochádzajúce z~často sa vyskytujúcich knižníc alebo jednoduchá modifikácia obrázkových súborov, ktorá môže detekciu znemožniť. Napriek týmto limitáciám má implementovaná metóda potenciál na odhalenie prebalených aplikácií. Od svojho spustenia v~marci 2018 zaznamenala 4\,000 požiadaviek na detekciu prebalených aplikácií. 9,8 \% analyzovaných aplikácií bolo vyhodnotených ako nebezpečné.

Spojením mobilnej aplikácie a serverovej časti vzniklo softvérové riešenie, umožňujúce detekciu prebalených aplikácií priamo v~Android zariadení. Obidve tieto časti boli nasadené v~produkčnom prostredí a sú dostupné užívateľom. Zdrojový kód aplikácií vytvorených v~rámci tejto práce je voľne prístupný.
