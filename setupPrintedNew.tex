%%%%%%%%%%%%%%%%%%%%%%%%%%%%%%%%%%%%%%%%%%%%%%%%%%%%%%%%%%%%%%%%%%%%
%% I, the copyright holder of this work, release this work into the
%% public domain. This applies worldwide. In some countries this may
%% not be legally possible; if so: I grant anyone the right to use
%% this work for any purpose, without any conditions, unless such
%% conditions are required by law.
%%%%%%%%%%%%%%%%%%%%%%%%%%%%%%%%%%%%%%%%%%%%%%%%%%%%%%%%%%%%%%%%%%%%

\documentclass[
  printed, %% This option enables the default options for the
           %% digital version of a document. Replace with `printed`
           %% to enable the default options for the printed version
           %% of a document.
  twoside, %% This option enables double-sided typesetting. Use at
           %% least 120 g/m² paper to prevent show-through. Replace
           %% with `oneside` to use one-sided typesetting; use only
           %% if you don’t have access to a double-sided printer,
           %% or if one-sided typesetting is a formal requirement
           %% at your faculty.
  table,   %% This option causes the coloring of tables. Replace
           %% with `notable` to restore plain LaTeX tables.
  lof,     %% This option prints the List of Figures. Replace with
           %% `nolof` to hide the List of Figures.
  lot,     %% This option prints the List of Tables. Replace with
           %% `nolot` to hide the List of Tables.
  %% More options are listed in the user guide at
  %% <http://mirrors.ctan.org/macros/latex/contrib/fithesis/guide/mu/fi.pdf>.
]{fithesis3}
%% The following section sets up the locales used in the thesis.
\usepackage[resetfonts]{cmap} %% We need to load the T2A font encoding
\usepackage[T1,T2A]{fontenc}  %% to use the Cyrillic fonts with Russian texts.
\usepackage[
  main=slovak, %% By using `czech` or `slovak` as the main locale
                %% instead of `english`, you can typeset the thesis
                %% in either Czech or Slovak, respectively.
  english, german, russian, slovak %% The additional keys allow
]{babel}        %% foreign texts to be typeset as follows:
%%
%%   \begin{otherlanguage}{german}  ... \end{otherlanguage}
%%   \begin{otherlanguage}{russian} ... \end{otherlanguage}
%%   \begin{otherlanguage}{czech}   ... \end{otherlanguage}
%%   \begin{otherlanguage}{slovak}  ... \end{otherlanguage}
%%
%% For non-Latin scripts, it may be necessary to load additional
%% fonts:
\usepackage{paratype}
\def\textrussian#1{{\usefont{T2A}{PTSerif-TLF}{m}{rm}#1}}
%%
%% The following section sets up the metadata of the thesis.
\thesissetup{
	date          = \the\year/\the\month/\the\day,
    university    = mu,
    faculty       = fi,
    type          = mgr,
    author        = Martin Styk,
    gender        = m,
    advisor       = {Ing. Mgr. et Mgr. Zdeněk Říha, Ph.D.},
    title         = {Detekcia prebalených aplikácií v OS Android},
    keywords      = {APK súbor, Android, malvér, prebalené aplikácie, analýza aplikácií, AndroidManifest.xml},
    abstract      = {Táto diplomová práca sa zameriava na detekciu prebalených aplikácií na platforme Android. Cieľom práce je vytvoriť systém detekcie prebalených aplikácií. V rámci práce je vytvorená mobilná aplikácia pre systém Android, ktorá analyzuje nainštalované aplikačné balíčky z ktorých získava detailné metadáta. 
Práca obsahuje návrh a implementáciu metódy detekcie prebalených aplikácií. Algoritmus detekcie využíva metadáta získané mobilnou aplikáciu. Implementovaná metóda je založená na zhode obrázkových súborov medzi viacerými aplikáciami. 

\noindent Teoretická časť práce sa zaoberá Android aplikáciami. Práca predstavuje metódu prebaľovania APK súborov a existujúce spôsoby detekcie takto modifikovaných aplikácií.

%This master's thesis concentrates on detection of repackaged Android applications. The aim of this thesis is to create repackaged applications detection system. As a part of this thesis android application which obtains detailed metadata about installed application packages is created.

%Thesis contains design and implementation of technique to detect repackaged applications. Based on obtained metadata, application detects repackaged malicious applications. Detection of repackaged applications is based on image similarity.

%The theoretical part discusses Android applications. Thesis introduces APK repackaging techniques and known methods of detection of repackaged applications.

%The aim of this master's thesis is to create Android application, which obtains metadata about installed application packages. Based on obtained metadata, application detects repackaged malicious applications. Detection of repackaged applications is based on image similarity.
%The theoretical part discusses Android applications. Thesis introduces APK repackaging techniques and known methods of detection of repackaged applications.},
    thanks        = {Rád by som sa poďakoval vedúcemu práce Ing. Mgr. et Mgr. Zdeňkovi Říhovi, Ph.D. za venovaný čas, ochotu a cenné pripomienky, ktoré mi pomohli pri tvorbe tejto práce.},
    bib           = bibliografie.bib,
}
\usepackage{makeidx}      %% The `makeidx` package contains
\makeindex                %% helper commands for index typesetting.
%% These additional packages are used within the document:
\usepackage{paralist} %% Compact list environments
\usepackage{amsmath}  %% Mathematics
\usepackage{amsthm}
\usepackage{amsfonts}
\usepackage{url}      %% Hyperlinks
\usepackage{markdown} %% Lightweight markup
\usepackage{listings} %% Source code highlighting
\lstset{
  basicstyle      = \ttfamily,%
  identifierstyle = \color{black},%
  keywordstyle    = \color{blue},%
  keywordstyle    = {[2]\color{cyan}},%
  keywordstyle    = {[3]\color{olive}},%
  stringstyle     = \color{teal},%
  captionpos      = b,%
  commentstyle    = \itshape\color{magenta}}
\usepackage{floatrow} %% Putting captions above tables
\floatsetup[table]{capposition=top}

\usepackage{color}
\definecolor{gray}{rgb}{0.4,0.4,0.4}
\definecolor{black}{rgb}{0,0,0}
\definecolor{light-gray}{rgb}{0,0,0}
\definecolor{darkblue}{rgb}{0.0,0.0,0.6}
\definecolor{cyan}{rgb}{0.0,0.6,0.6}

\lstdefinelanguage{XML}
{
  morestring=[b]",
  morestring=[s]{>}{<},
  morecomment=[s]{<?}{?>},
  stringstyle=\color{black},
  identifierstyle=\color{darkblue},
  keywordstyle=\color{cyan},
  morekeywords={xmlns,version,type}% list your attributes here
}
\renewcommand{\lstlistingname}{Kód}


%% These additional packages are used within the document:\usepackage{menukeys}
\usepackage{pgf-pie}
\usepackage{color}
\usepackage{tikz}
\usepackage{longtable}
\usepackage{booktabs}
\usepackage{floatrow}
\usepackage[font=small,labelfont=bf]{caption} % Required for specifying captions to tables and figures

\newcommand{\zv}{\textit}
\newcommand{\cesta}{\zv}
\newcommand{\bod}{\zv}
